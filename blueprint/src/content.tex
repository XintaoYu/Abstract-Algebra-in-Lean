\title{Abstract Algebra in Lean}


%\home{https://XintaoYu.github.io/blueprint-template/}
%\github{https://github.com/XintaoYu/Abstract-Algebra-in-Lean}
%\dochome{https://reaslab.github.io/blueprint-template/doc/}

% \home{localhost:8080}
% \dochome{localhost:8080/doc}

\maketitle


\tableofcontents
\section{Introduction}



\section{Exercise}

% To avoid bibtex errors
\nocite{*} % Delete this line if you have citations.

\begin{theorem}[Exercise 1]\label{Ex1}
  \leanok
    Suppose that $\star$ is an associative binary operation on a set $S$. Let
    \[
    H = \{a \in S \mid a \star x = x \star a \text{ for all } x \in S\}.
    \]
    Show that $H$ is closed under $\star$. (We think of $H$ as consisting of all elements of $S$ that commute with every element in $S$.)
\end{theorem}
\begin{proof}
  \leanok
\end{proof}


\begin{theorem}[Exercise 2]\label{Ex2}
  \leanok
    Are all groups of 4 elements commutative?
\end{theorem}
\begin{proof}
  \leanok
\end{proof}


\begin{theorem}[Exercise 3]\label{Ex3}
  \leanok
    Show that if $G$ is a finite group with identity $e$ and with an even number of elements, then there is $a \neq e$ in $G$ such that $a \star a = e$.
\end{theorem}
\begin{proof}
  \leanok
\end{proof}


\begin{theorem}[Exercise 4]\label{Ex4}
  \leanok
    Show that if $H$ and $K$ are subgroups of an abelian group $G$, then
    \[
    \{hk \mid h \in H \text{ and } k \in K\}
    \]
    is a subgroup of $G$.
\end{theorem}
\begin{proof}
  \leanok
\end{proof}

\begin{theorem}[Exercise 5]\label{Ex5}
  \leanok
Show that a group with no proper nontrivial subgroups is cyclic.

\end{theorem}
\begin{proof}
  \leanok
\end{proof}

\begin{theorem}[Exercise 6]\label{Ex6}
  \leanok
   Let $p$ be a prime number. Find the number of generators of the cyclic group $\mathbb{Z}_{p^r}$, where $r$ is an integer $\geq 1$.
\end{theorem}
\begin{proof}
  \leanok
\end{proof}

\begin{theorem}[Exercise 7]\label{Ex7}
    Let $G$ be an abelian group and let $H$ and $K$ be finite cyclic subgroups with $|H| = r$ and $|K| = s$.
    \begin{enumerate}
        \item[a)] Show that if $r$ and $s$ are relatively prime, then $G$ contains a cyclic subgroup of order $rs$.

        \item[b)] Generalizing part (a), show that $G$ contains a cyclic subgroup of order the least common multiple of $r$ and $s$.
    \end{enumerate}
\end{theorem}
\begin{proof}

\end{proof}

\begin{theorem}[Exercise 8]\label{Ex8}
    Show that for $n \geq 3$, there exists a nonabelian group with $2n$ elements that is generated by two elements of order 2.
\end{theorem}
\begin{proof}

\end{proof}

\begin{theorem}[Exercise 9]\label{Ex9}
    Find the number of elements in the set $\{\sigma \in S_4 \mid \sigma(3) = 3\}$.
\end{theorem}
\begin{proof}

\end{proof}

\begin{theorem}[Exercise 10]\label{Ex10}
    Show that $S_n$ is a nonabelian group for $n \geq 3$.
\end{theorem}
\begin{proof}

\end{proof}

\begin{theorem}[Exercise 11]\label{Ex11}
    Let $G$ be a group. Prove that the permutations $\rho_a : G \to G$, where $\rho_a(x) = xa$ for $a \in G$ and $x \in G$, do form a group isomorphic to $G$.
\end{theorem}
\begin{proof}

\end{proof}

\begin{theorem}[Exercise 12]\label{Ex12}
\leanok
    Let $G$ be a group and let $a$ be a fixed element of $G$. Show that the map $\lambda_a : G \to G$, given by $\lambda_a(g) = ag$ for $g \in G$, is a permutation of the set $G$.
\end{theorem}
\begin{proof}
    \leanok
\end{proof}

\begin{theorem}[Exercise 13]\label{Ex13}
    Show that $S_n$ is generated by $\{(1, 2), (1, 2, 3, \ldots, n)\}$. [Hint: Show that as $r$ varies, $(1, 2, 3, \ldots, n)^r(1, 2)(1, 2, 3, \ldots, n)^{n-r}$ gives all the transpositions $(1, 2), (2, 3), (3, 4), \ldots, (n-1, n), (n, 1)$. Then show that any transposition is a product of some of these transpositions and use Corollary 9.12.]
\end{theorem}
\begin{proof}

\end{proof}

\begin{theorem}[Exercise 14]\label{Ex14}
    Let $G$ be a group of order $pq$, where $p$ and $q$ are prime numbers. Show that every proper subgroup of $G$ is cyclic.
\end{theorem}
\begin{proof}

\end{proof}

\begin{theorem}[Exercise 15]\label{Ex15}
    Show that a finite cyclic group of order $n$ has exactly one subgroup of each order $d$ dividing $n$, and that these are all the subgroups it has.
\end{theorem}
\begin{proof}

\end{proof}

\begin{theorem}[Exercise 16]\label{Ex16}
    Let $H$ and $K$ be groups and let $G = H \times K$. Recall that both $H$ and $K$ appear as subgroups of $G$ in a natural way. Show that these subgroups $H$ (actually $H \times \{e\}$) and $K$ (actually $\{e\} \times K$) have the following properties:
    \begin{enumerate}
        \item[a)] Every element of $G$ is of the form $hk$ for some $h \in H$ and $k \in K$.

        \item[b)] $hk = kh$ for all $h \in H$ and $k \in K$.

        \item[c)] $H \cap K = \{e\}$.
    \end{enumerate}
\end{theorem}
\begin{proof}

\end{proof}

\begin{theorem}[Exercise 17]\label{Ex17}
    Let $H$ and $K$ be subgroups of a group $G$ satisfying the three properties listed in the preceding exercise. Show that for each $g \in G$, the expression $g = hk$ for $h \in H$ and $k \in K$ is unique. Then let each $g$ be renamed $(h, k)$. Show that, under this renaming, $G$ becomes structurally identical (isomorphic) to $H \times K$.
\end{theorem}
\begin{proof}

\end{proof}

\begin{theorem}[Exercise 18]\label{Ex18}
    Show that a finite abelian group is not cyclic if and only if it contains a subgroup isomorphic to $\mathbb{Z}_p \times \mathbb{Z}_p$ for some prime $p$.
\end{theorem}
\begin{proof}

\end{proof}

\begin{theorem}[Exercise 19]\label{Ex19}
    Example: Let $S_n$ be the symmetric group on $n$ letters, and let $\Phi : S_n \to \mathbb{Z}_2$ be defined by $\Phi(\sigma) = 0$ if $\sigma$ is an even permutation, 1 if $\sigma$ is an odd permutation. Show that $\Phi$ is a homomorphism.

    Solution: We must show that $\Phi(\sigma\mu) = \Phi(\sigma) + \Phi(\mu)$ for all choices of $\sigma, \mu \in S_n$. Note that the operation on the right-hand side of this equation is written additively since it takes place in the group $\mathbb{Z}_2$. Verifying this equation amounts to checking just four cases:
    \begin{enumerate}
        \item $\sigma$ odd and $\mu$ odd,
        \item $\sigma$ odd and $\mu$ even,
        \item $\sigma$ even and $\mu$ odd,
        \item $\sigma$ even and $\mu$ even.
    \end{enumerate}
\end{theorem}
\begin{proof}

\end{proof}
